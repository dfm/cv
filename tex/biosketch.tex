% The formatting of this CV is based on @davidwhogg's layout.

\documentclass[12pt,letterpaper]{article}

\usepackage{color}
\usepackage{fancyhdr}
\usepackage{hyperref}
\usepackage{ifthen}

% \usepackage[yyyymmdd]{datetime}
% \renewcommand{\dateseparator}{-}

% Link formatting.
\definecolor{numcolor}{rgb}{0.5,0.5,0.5}
\definecolor{linkcolor}{rgb}{0,0,0.4}
\hypersetup{%
    colorlinks=true,        % false: boxed links; true: colored links
    linkcolor=linkcolor,    % color of internal links
    citecolor=linkcolor,    % color of links to bibliography
    filecolor=linkcolor,    % color of file links
    urlcolor=linkcolor      % color of external links
}

% Text formatting.
\newcommand{\foreign}[1]{\textit{#1}}
\newcommand{\etal}{\foreign{et~al.}}
\newcommand{\project}[1]{\textsl{#1}}
\definecolor{grey}{rgb}{0.5,0.5,0.5}
\newcommand{\deemph}[1]{\textcolor{grey}{\footnotesize{#1}}}

% literature links--use doi if you can
  \newcommand{\doi}[2]{\emph{\href{http://dx.doi.org/#1}{{#2}}}}
  \newcommand{\ads}[2]{\href{http://adsabs.harvard.edu/abs/#1}{{#2}}}
  \newcommand{\isbn}[1]{{\footnotesize(\textsc{isbn:}{#1})}}
  \newcommand{\arxiv}[1]{{\href{http://arxiv.org/abs/#1}{arXiv:{#1}}}}

% Section headings.
\newcommand{\cvheading}[1]{\addvspace{1ex}\pagebreak[2]\par\textbf{#1}\nopagebreak\vspace{-0.4em}}

% Set up the custom unordered list.
\newcounter{refpubnum}
\newcommand{\cvlist}{%
    \rightmargin=0in
    \leftmargin=0.15in
    \topsep=0ex
    \partopsep=0pt
    \itemsep=0.2ex
    \parsep=0pt
    \itemindent=-1.0\leftmargin
    \listparindent=0.0\leftmargin
    \settowidth{\labelsep}{~}
    \usecounter{refpubnum}
}

% Margins and spaces.
\raggedright
\setlength{\oddsidemargin}{0in}
\setlength{\topmargin}{0in}
\setlength{\headsep}{0.20in}
\setlength{\headheight}{0.25in}
\setlength{\textheight}{9.1in}
\addtolength{\topmargin}{-\headsep}
\addtolength{\topmargin}{-\headheight}
\setlength{\textwidth}{6.50in}
\setlength{\parindent}{0in}
\setlength{\parskip}{1ex}

% Headings and footings.
\renewcommand{\headrulewidth}{0pt}
\pagestyle{fancy}
\lhead{\deemph{Daniel Foreman-Mackey}}
\chead{\deemph{Curriculum Vitae}}
\rhead{\deemph{\thepage}}
\cfoot{\deemph{Last updated: \today}}

% Journal names.
\newcommand{\aj}{AJ}
\newcommand{\apj}{ApJ}
\newcommand{\pasp}{PASP}
\newcommand{\mnras}{MNRAS}


\begin{document}\thispagestyle{empty}\sloppy\sloppypar\raggedbottom

\textbf{\Large Daniel Foreman-Mackey} \hfill
\textsf{\small foreman.mackey@gmail.com, http://dfm.io} \\[0.5ex]
Associate Research Scientist, Center for Computational Astronomy, Flatiron Institute\\[0.5ex]

\cvheading{Professional preparation}
\begin{list}{}{\cvlist}
\item
2017--, Associate Research Scientist, Flatiron Institute.
\item
2015--2017, Sagan Postdoctoral Fellow, University of Washington.
\item
PhD 2015, Department of Physics, New York University. Advisor: Hogg
\item
MSc 2010, Department of Physics, Queen's University, Canada. Advisor: Widrow
\item
BSc 2008, Department of Physics, McGill University, Canada.
\end{list}

\cvheading{Selected publications}
\begin{list}{}{\cvlist}
\item[{\color{numcolor}\scriptsize6}] \textbf{Foreman-Mackey, Daniel}; Morton, Timothy D.; Hogg, David W.; Agol, Eric; \& Sch{\"o}lkopf, Bernhard, 2016, \doi{10.3847/0004-6256/152/6/206}{The Population of Long-period Transiting Exoplanets}, \aj, \textbf{152}, 206 (\arxiv{1607.08237}) [\href{https://ui.adsabs.harvard.edu/abs/2016AJ....152..206F}{63 citations}]

\item[{\color{numcolor}\scriptsize5}] \textbf{Foreman-Mackey, Daniel}, 2016, \doi{10.21105/joss.00024}{corner.py: Scatterplot matrices in Python}, The Journal of Open Source Software, \textbf{1}, 2 [\href{https://scholar.google.com/scholar?cites=1835087844145558435,17836006976722650130,17325274697099535179,14220488595059618709,12820425635803494730,7284810048757141243,16003095824317425768}{988 citations}]

\item[{\color{numcolor}\scriptsize4}] Montet, Benjamin T.; Morton, Timothy D.; \textbf{Foreman-Mackey, Daniel}; Johnson, John Asher; \etal, 2015, \doi{10.1088/0004-637X/809/1/25}{Stellar and Planetary Properties of K2 Campaign 1 Candidates and Validation of 17 Planets, Including a Planet Receiving Earth-like Insolation}, \apj, \textbf{809}, 25 (\arxiv{1503.07866}) [\href{https://ui.adsabs.harvard.edu/abs/2015ApJ...809...25M}{97 citations}]

\item[{\color{numcolor}\scriptsize3}] \textbf{Foreman-Mackey, Daniel}; Montet, Benjamin T.; Hogg, David W.; Morton, Timothy D.; \etal, 2015, \doi{10.1088/0004-637X/806/2/215}{A Systematic Search for Transiting Planets in the K2 Data}, \apj, \textbf{806}, 215 (\arxiv{1502.04715}) [\href{https://ui.adsabs.harvard.edu/abs/2015ApJ...806..215F}{99 citations}]

\item[{\color{numcolor}\scriptsize2}] \textbf{Foreman-Mackey, Daniel}; Hogg, David W.; \& Morton, Timothy D., 2014, \doi{10.1088/0004-637X/795/1/64}{Exoplanet Population Inference and the Abundance of Earth Analogs from Noisy, Incomplete Catalogs}, \apj, \textbf{795}, 64 (\arxiv{1406.3020}) [\href{https://ui.adsabs.harvard.edu/abs/2014ApJ...795...64F}{181 citations}]

\item[{\color{numcolor}\scriptsize1}] \textbf{Foreman-Mackey, Daniel}; Hogg, David W.; Lang, Dustin; \& Goodman, Jonathan, 2013, \doi{10.1086/670067}{emcee: The MCMC Hammer}, \pasp, \textbf{125}, 306 (\arxiv{1202.3665}) [\href{https://ui.adsabs.harvard.edu/abs/2013PASP..125..306F}{5018 citations}]
\end{list}

\cvheading{Popular open-source software}
\begin{list}{}{\cvlist}

\item {\bf emcee} ---
    MCMC sampling in Python. Popular in astronomy;
    the paper has over 1000 citations.
    \url{emcee.readthedocs.io}

\item {\bf george} ---
    Blazingly fast Gaussian processes for regression. Implemented in C++ and
    Python bindings. Joint work with applied mathematicians at NYU.
    \url{george.readthedocs.io}

\item {\bf celerite} ---
    Scalable computations for Gaussian process regression for one-dimensional
    problems.
    \url{celerite.readthedocs.io}

\item {\bf corner.py} ---
    Simple corner plots (or scatterplot matrices) in Python.
    \url{corner.readthedocs.io}

\end{list}

%\cvheading{Honors}
%\begin{list}{}{\cvlist}
%\item Kavli Fellow, 2015.
%\item Sagan Postdoctoral Fellowship, 2015--present.
%\item James Arthur Graduate Fellowship, 2014.
%\item Horizon Fellowship in the Natural \& Physical Sciences, 2012.
%\item Henry M. MacCracken Fellowship, 2010.
%\item NSERC Undergraduate Summer Research Award, 2007.
%\end{list}

\end{document}
